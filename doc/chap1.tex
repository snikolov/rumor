%% This is an example first chapter.  You should put chapter/appendix that you
%% write into a separate file, and add a line \include{yourfilename} to
%% main.tex, where `yourfilename.tex' is the name of the chapter/appendix file.
%% You can process specific files by typing their names in at the 
%% \files=
%% prompt when you run the file main.tex through LaTeX.
\chapter{Introduction}
Detection, classification, and prediction of events in timeseries is a ubiquitous problem. From detecting malfunctions in a production plant, to predicting an imminent market crash, to revealing emerging popular topics in a social network, timeseries analysis methods (????) are fundamental for extracting meaning from any time-varying data.

In recent years, there has been an explosion of data from virtually every human endeavor --- including healthcare, biology, physics, finance, and social interaction online and offline --- that demands for us to extract the valuable insights hidden within it. At the same time, developments in distributed computation technologies have made it easier than ever to exploit the structure in this data to do inference at scale.

\section{Previous Work}
Timeseries classification. Event detection. Network cascade models. Branching process models.

\section{My Approach}
Simple models prove ineffective when exposed to the vast variety of real world situations. (What about adaptive models?) I propose a nonparametric framework for doing inference on timeseries (VAGUE). In this model, we posit that there are latent timeseries, each corresponding to a prototypical event of a certain type, and each observed timeseries is a noisy observation of one of the latent timeseries.

How does this apply to 1) Detection 2) Classification 3) Prediction? What's the difference between classification and detection?

\subsection{Anomaly Detection}
A timeseires in a given window is compared to a bundle of timeseries that represents ``normal'' events. We compute the probability that the observed timeseries was generated by one of the latent timeseries in the bundle and declare observations with low probability to be anomalies.

\subsection{Classification}
A timeseries in a given window is compared to two {\em reference} timeseries bundles --- one consisting of positive examples and the other of negative examples. We want to find out whether it is more likely that the observed timeseries was generated by one of the latent timeseries in the positive bundle or one of the latent timeseries in the negative bundle.

\subsection{Prediction} A timeseries in a given window is compared to a single reference bundle of timeseries. We want to find out the probability that the observed timeseries was generated from the same latent timeseries as each of the timeseries in the bundle. We then compute the most likely change in the observed timeseries by observing the changes in each reference timeseries and weighting the change by the probability that the observed timeseries and the reference timeseries were generated by the same latent timeseries.
 
************Lots of work in clustering, but this has an explicit probabilistic model.

\subsection{Application: Detecting Outbreaks of Popular Topics on Twitter}
As an application, I apply the method (CALL IT SOMETHING) to the problem of detecting emerging popular topics (called {\em trends}) on Twitter. I use a set of labeled examples of timeseries correspondingn to topics that eventually became trending topics and topics that did not. I then perform online classification of a given topic to label it as trending or not trending at a particular time.

***********Talk about results and main contributions.
