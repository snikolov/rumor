\documentclass[10pt]{article}

\title{A Nonparametric Method for Early Detection of Trending
  Topics} \author{ Stanislav Nikolov$^{\dagger,*}$ and Devavrat Shah$^\dagger$\\
  \normalsize \{snikolov,devavrat\}@mit.edu\\
  \normalsize $^\dagger$Department of EECS, Massachusetts Institute of Technology\\
  \normalsize $^*$Twitter Inc.}

\usepackage[margin=0.5in]{geometry}
\date{}

\begin{document}

\maketitle

\begin{abstract}
Online social networks can be used as networks of human sensors to
detect important events \cite{Zhao} --- from a global breaking news
story to a fire down the street. It is important to be able to
detect such events as early as possible. We propose a nonparametric
method that predicts {\em trending topics} on Twitter by comparing a
recent activity signal for a topic to a large collection of
historical signals for trending and non-trending topics. We posit
that the signals observed for each class of topics were
generated by an unknown set of {\em latent source} signals for that class
according to a stochastic model depending on the {\em distance}
between the observation and its latent source. Using our stochastic
model, we derive a class estimator based on the ratio of conditional
class probabilities. We are able to detect trends in advance of
Twitter 79\% of the time, with a mean early advantage of 1 hour and 26
minutes, while maintaining a true positive rate of 95\% and a false
positive rate of 4\%. Our method allows for tradeoffs between TPR,
FPR, and relative detection time, scales to large amounts of data, and
provides a broadly applicable framework for nonparametric
classification.
\end{abstract}

\subsubsection*{Empirical Observations}
% Notes: 
% * Twitter insists that ``Tweet'' is capitalized.
% * I am only allowed to reproduce the text of my own Tweets.  

On Twitter, users can post messages known as {\em Tweets}. There are
over 400 million Tweets written per day, many of which can be
considered {\em about} one or more topics. For example, this tweet by
one of the authors (Twitter handle @snikolov) {\em ``Stuyvesant High
  School Taps `Stuy Mafia' at Google, Foursquare to Enhance Computer
  Science Program via @Betabeat http://betabeat.com...''} is about
``Stuyvesant High School'', ``Computer Science'', and so on. Some
topics gain sufficient popularity and start {\em trending} i.e. they
are featured on a list of top ten trending topics on Twitter. We
observe that trending topics are distinguished from nontrending topics
by their pattern of activity leading up to the time the topic is
detected by Twitter (the true onset). In particular, the activity of a
soon to be trending topic is characterized by frequent sharp jumps of
high magnitude over some baseline activity. Let $\rho[n]$ be the
discrete derivative at time step $n$ of the volume $v(t)$ of Tweets
about a topic over time, for $n = 1, \dots, N$. We shall call this the
rate of Tweets. We construct a baseline-normalized rate $\rho_b[n] =
\left(\rho[n]/b \right)^{\beta}$ by removing the baseline rate $b =
(\sum_{n} \rho[n])/N$

Timeseries activity is incredibly diverse. Rather than training
a model to distinguish between trending and non trending topics, which
would require us to assume a model structure, we propose a
nonparametric method that relies directly on large amounts of data as
a proxy for the latent model structure


\subsubsection*{Data Model}
We posit that all observed signals were generated by an unknown set of
{\em latent source} signals. A

Stochastic model

Class Probability propto ...etc. 

All we have to do is compute distances to examples. 

We can do it in parallel.

\subsubsection*{Results}
Figure with example early detection (> 2 hrs early)
Figure with Distribution of early and late in best case
Figure with Tradeoffs (ROC curve/envelope and distribution of early/late at two or three points)

\subsubsection*{Conclusion}
Broadly applicable

\begin{footnotesize}
\bibliographystyle{plain}
\bibliography{trend}
\end{footnotesize}
\end{document}
